\documentclass[a4paper,12pt,twoside]{memoir}

% Castellano
\usepackage[spanish,es-tabla]{babel}
\selectlanguage{spanish}
\usepackage[utf8]{inputenc}
\usepackage[T1]{fontenc}
\usepackage{lmodern} % Scalable font
\usepackage{microtype}
\usepackage{placeins}
\usepackage{url} % Mejora el tratamiento de URL, p. ej. en saltos de línea.

% Bibliografía bajo norma UNE-ISO 690:2013.
\usepackage[
  backend=biber,     % "Backend" que genera la bibliografía.
  style=iso-numeric, % Usa el método de referencias numéricas.
  autolang=other,    % Soporta múltiples lenguajes en la bibliografía.
  sortlocale=es_ES,  % Ordena la bibliografía usando el lenguaje indicado.
  bibencoding=UTF8   % Codificación de caracteres de los archivos .bib.
]{biblatex}

% Indica el archivo usado por la memoria.
\addbibresource{bibliografia.bib}

\RequirePackage{booktabs}
\RequirePackage[table]{xcolor}
\RequirePackage{xtab}
\RequirePackage{multirow}

% Links
\usepackage[colorlinks]{hyperref}
\hypersetup{
	allcolors = {red}
}

% Ecuaciones
\usepackage{amsmath}

% Rutas de fichero / paquete
\newcommand{\ruta}[1]{{\sffamily #1}}

% Párrafos
\nonzeroparskip

% Imágenes
\usepackage{graphicx}
\newcommand{\imagen}[2]{
	\begin{figure}[!h]
		\centering
		\includegraphics[width=0.9\textwidth]{#1}
		\caption{#2}\label{fig:#1}
	\end{figure}
	\FloatBarrier
}

\newcommand{\imagenflotante}[2]{
	\begin{figure}%[!h]
		\centering
		\includegraphics[width=0.9\textwidth]{#1}
		\caption{#2}\label{fig:#1}
	\end{figure}
}

% El comando \figura nos permite insertar figuras cómodamente, y utilizando
% siempre el mismo formato. Los parámetros son:
% 1 -> Porcentaje del ancho de página que ocupará la figura (de 0 a 1)
% 2 --> Fichero de la imagen
% 3 --> Texto a pie de imagen
% 4 --> Etiqueta (label) para referencias
% 5 --> Opciones que queramos pasar a \includegraphics
% 6 --> Opciones de posicionamiento a pasar a \begin{figure}
\newcommand{\figuraConPosicion}[6]{%
  \setlength{\anchoFloat}{#1\textwidth}%
  \addtolength{\anchoFloat}{-4\fboxsep}%
  \setlength{\anchoFigura}{\anchoFloat}%
  \begin{figure}[#6]
    \begin{center}%
      \Ovalbox{%
        \begin{minipage}{\anchoFloat}%
          \begin{center}%
            \includegraphics[width=\anchoFigura,#5]{#2}%
            \caption{#3}%
            \label{#4}%
          \end{center}%
        \end{minipage}
      }%
    \end{center}%
  \end{figure}%
}

%
% Comando para incluir imágenes en formato apaisado (sin marco).
\newcommand{\figuraApaisadaSinMarco}[5]{%
  \begin{figure}%
    \begin{center}%
    \includegraphics[angle=90,height=#1\textheight,#5]{#2}%
    \caption{#3}%
    \label{#4}%
    \end{center}%
  \end{figure}%
}
% Para las tablas
\newcommand{\otoprule}{\midrule [\heavyrulewidth]}
%
% Nuevo comando para tablas pequeñas (menos de una página).
\newcommand{\tablaSmall}[5]{%
 \begin{table}[h]
  \begin{center}
   \rowcolors {2}{gray!35}{}
   \begin{tabular}{#2}
    \toprule
    #4
    \otoprule
    #5
    \bottomrule
   \end{tabular}
   \caption{#1}
   \label{tabla:#3}
  \end{center}
 \end{table}
}

%
% Nuevo comando para tablas pequeñas (menos de una página).
\newcommand{\tablaSmallSinColores}[5]{%
 \begin{table}[H]
  \begin{center}
   \begin{tabular}{#2}
    \toprule
    #4
    \otoprule
    #5
    \bottomrule
   \end{tabular}
   \caption{#1}
   \label{tabla:#3}
  \end{center}
 \end{table}
}

\newcommand{\tablaApaisadaSmall}[5]{%
\begin{landscape}
  \begin{table}
   \begin{center}
    \rowcolors {2}{gray!35}{}
    \begin{tabular}{#2}
     \toprule
     #4
     \otoprule
     #5
     \bottomrule
    \end{tabular}
    \caption{#1}
    \label{tabla:#3}
   \end{center}
  \end{table}
\end{landscape}
}

%
% Nuevo comando para tablas grandes con cabecera y filas alternas coloreadas en gris.
\newcommand{\tabla}[6]{%
  \begin{center}
    \tablefirsthead{
      \toprule
      #5
      \otoprule
    }
    \tablehead{
      \multicolumn{#3}{l}{\small\sl continúa desde la página anterior}\\
      \toprule
      #5
      \otoprule
    }
    \tabletail{
      \hline
      \multicolumn{#3}{r}{\small\sl continúa en la página siguiente}\\
    }
    \tablelasttail{
      \hline
    }
    \bottomcaption{#1}
    \rowcolors {2}{gray!35}{}
    \begin{xtabular}{#2}
      #6
      \bottomrule
    \end{xtabular}
    \label{tabla:#4}
  \end{center}
}

%
% Nuevo comando para tablas grandes con cabecera.
\newcommand{\tablaSinColores}[6]{%
  \begin{center}
    \tablefirsthead{
      \toprule
      #5
      \otoprule
    }
    \tablehead{
      \multicolumn{#3}{l}{\small\sl continúa desde la página anterior}\\
      \toprule
      #5
      \otoprule
    }
    \tabletail{
      \hline
      \multicolumn{#3}{r}{\small\sl continúa en la página siguiente}\\
    }
    \tablelasttail{
      \hline
    }
    \bottomcaption{#1}
    \begin{xtabular}{#2}
      #6
      \bottomrule
    \end{xtabular}
    \label{tabla:#4}
  \end{center}
}

%
% Nuevo comando para tablas grandes sin cabecera.
\newcommand{\tablaSinCabecera}[5]{%
  \begin{center}
    \tablefirsthead{
      \toprule
    }
    \tablehead{
      \multicolumn{#3}{l}{\small\sl continúa desde la página anterior}\\
      \hline
    }
    \tabletail{
      \hline
      \multicolumn{#3}{r}{\small\sl continúa en la página siguiente}\\
    }
    \tablelasttail{
      \hline
    }
    \bottomcaption{#1}
  \begin{xtabular}{#2}
    #5
   \bottomrule
  \end{xtabular}
  \label{tabla:#4}
  \end{center}
}



\definecolor{cgoLight}{HTML}{EEEEEE}
\definecolor{cgoExtralight}{HTML}{FFFFFF}

%
% Nuevo comando para tablas grandes sin cabecera.
\newcommand{\tablaSinCabeceraConBandas}[5]{%
  \begin{center}
    \tablefirsthead{
      \toprule
    }
    \tablehead{
      \multicolumn{#3}{l}{\small\sl continúa desde la página anterior}\\
      \hline
    }
    \tabletail{
      \hline
      \multicolumn{#3}{r}{\small\sl continúa en la página siguiente}\\
    }
    \tablelasttail{
      \hline
    }
    \bottomcaption{#1}
    \rowcolors[]{1}{cgoExtralight}{cgoLight}

  \begin{xtabular}{#2}
    #5
   \bottomrule
  \end{xtabular}
  \label{tabla:#4}
  \end{center}
}

\graphicspath{ {./img/} }

% Capítulos
\chapterstyle{bianchi}
\newcommand{\capitulo}[2]{
	\setcounter{chapter}{#1}
	\setcounter{section}{0}
	\chapter*{#2}
	\addcontentsline{toc}{chapter}{#2}
	\markboth{#2}{#2}
}

% Apéndices
\renewcommand{\appendixname}{Apéndice}
\renewcommand*\cftappendixname{\appendixname}

\newcommand{\apendice}[1]{
	%\renewcommand{\thechapter}{A}
	\chapter{#1}
}

\renewcommand*\cftappendixname{\appendixname\ }

% Formato de portada
\makeatletter
\usepackage{xcolor}
\newcommand{\tutor}[1]{\def\@tutor{#1}}
\newcommand{\course}[1]{\def\@course{#1}}
\definecolor{cpardoBox}{HTML}{E6E6FF}
\def\maketitle{
  \null
  \thispagestyle{empty}
  % Cabecera ----------------
\noindent\includegraphics[width=\textwidth]{cabecera}\vspace{1cm}%
  \vfill
  % Título proyecto y escudo informática ----------------
  \colorbox{cpardoBox}{%
    \begin{minipage}{.8\textwidth}
      \vspace{.5cm}\Large
      \begin{center}
      \textbf{TFM del Máster Universitario en Ingeniería Informática}\vspace{.6cm}\\
      \textbf{\LARGE\@title{}}
      \end{center}
      \vspace{.2cm}
    \end{minipage}

  }%
  \hfill\begin{minipage}{.20\textwidth}
    \includegraphics[width=\textwidth]{escudoInfor}
  \end{minipage}
  \vfill
  % Datos de alumno, curso y tutores ------------------
  \begin{center}%
  {%
    \noindent\LARGE
    Presentado por \@author{}\\ 
    en Universidad de Burgos --- \@date{}\\
    Tutor: \@tutor{}\\
  }%
  \end{center}%
  \null
  \cleardoublepage
  }
\makeatother

\newcommand{\nombre}{RPC} %%% cambio de comando
     
% Comando para formatear palabras procedentes de otros
% lenguajes distintos del castellano.
% Estas palabras se pueden escribir en letra cursiva 
\newcommand{\extranjerismo}[1]{\textit{#1}}
% o con comillas si no se dispone de cursiva.
%\newcommand{\extranjerismo}[1]{"{#1}"}

% Comando para formatear los títulos de otras obras de creación.
\newcommand{\titulo}[1]{\textit{#1}}

% Datos de portada
\title{Comunicación TCP/IP con sistemas empotrados}
\author{\nombre}
\tutor{AMG}
\date{\today}

\begin{document}

\maketitle

\newpage\null\thispagestyle{empty}

%%%%%%%%%%%%%%%%%%%%%%%%%%%%%%%%%%%%%%%%%%%%%%%%%%%%%%%%%%%%%%%%%%%%%%%%%%%%%%%%%%%%%%%%
\thispagestyle{empty}


\noindent\includegraphics[width=\textwidth]{cabecera}\vspace{1cm}

\noindent D. AMG, profesor del Departamento de Ingeniería Electromecánica, Área de Ingeniería de Sistemas y Automática.

\noindent Expone:

\noindent Que el alumno D. \nombre, con DNI 12345678Z, ha realizado el Trabajo final de máster del Máster Universitario en Ingeniería Informática titulado \titulo{Comunicación TCP/IP con sistemas empotrados}.

\noindent Y que dicho trabajo ha sido realizado por el alumno bajo la dirección del que suscribe, en virtud de lo cual se autoriza su presentación y defensa.

\begin{center} %\large
En Burgos, {\large \today}
\end{center}

\vfill\vfill\vfill

% Author and supervisor
%\begin{minipage}{0.45\textwidth}
%\begin{flushleft} %\large
%Vº. Bº. del Tutor:\\[2cm]
%D. nombre tutor
%\end{flushleft}
%\end{minipage}
%\hfill
%\begin{minipage}{0.45\textwidth}
%\begin{flushleft} %\large
%Vº. Bº. del co-tutor:\\[2cm]
%D. nombre co-tutor
%\end{flushleft}
%\end{minipage}
%\hfill
%
%\vfill

% para casos con solo un tutor comentar lo anterior
% y descomentar lo siguiente
Vº. Bº. del Tutor:\\[2cm]
D. AMG

\newpage\null\thispagestyle{empty}

\frontmatter

% Abstract en castellano
\renewcommand*\abstractname{Resumen}
\begin{abstract}
  Las placas de desarrollo facilitan el estudio y desarrollo de sistemas 
  empotrados. Un sistema empotrado conectado a una red de comunicaciones de
  datos obtiene una nueva vía de interacción con otros sistemas, ya sean
  empotrados o convencionales. Un sistema empotrado conectado permite
  interactuar de forma remota con él, pudiendo realizar entre otras operaciones
  la consulta de sus sensores o la activación de sus actuadores.

  En este proyecto se muestra como conectar una placa de desarrollo FRDM-K64F
  a una red de área local usando el conjunto de protocolos TCP/IP.
  Aprovechando tanto el \extranjerismo{hardware} del que dispone la placa como
  del \extranjerismo{hardware} conectado a ella, se ejemplifica la interacción
  con algunos de sus dispositivos a través de una aplicación web. Como
  dispositivos configurados para su uso remoto se encuentran el led integrado
  en la placa, una pantalla LCD y una placa de expansión.
\end{abstract}

\renewcommand*\abstractname{Descriptores}
\begin{abstract}
  Sistemas embebidos, placa desarrollo, familia de protocolos de internet,
  aplicación web.
\end{abstract}

\clearpage

% Abstract en inglés
\renewcommand*\abstractname{Abstract}
\begin{abstract}
  Development boards facilitate the study and development of embedded systems.
  An embedded system connected to a data network obtains a new way of 
  interaction with other systems, whether embedded or conventional. A connected
  embedded system allow to interact remotely with it, being able to carry out, 
  among other operations, the query of its sensors or the activation of its
  actuators.
  
  This project shows how to connect a FRDM-K64F development board to a local
  area network using the TCP / IP protocol suite. Taking advantage of both
  the hardware available on the board and the hardware connected to it, 
  the interaction with some of its devices is exemplified through a web
  application. As devices configured for remote use are the LED integrated in
  the board, an LCD screen and an expansion board.
\end{abstract}

\renewcommand*\abstractname{Keywords}
\begin{abstract}
  Embedded systems, development board, Internet protocol suite,
  web application.
\end{abstract}

\clearpage

% Índices
\tableofcontents

\clearpage

\listoffigures

\clearpage

\listoftables
\clearpage

\mainmatter
\capitulo{1}{Introducción}\label{ch:introduccion}
Los sistemas empotrados o embebidos (SE) son sistemas diseñados para realiza una
función específica y, por tanto, solo realizan una o unas pocas tareas
concretas. Este hecho les diferencia de otros sistemas de propósito general como
ordenadores o teléfonos móviles, que son capaces de realizar multitud de tares
de diferente naturaleza. 

Como el propio nombre de los SE indica, este tipo de sistemas se suele encontrar
integrado en otros sistemas eléctricos o mecánicos de mayor envergadura y que se
encargan de controlar. Por esta razón, se requiere que los SE cuenten con 
ciertas características: tamaño reducido, bajo consumo o bajo coste. Requisitos
que a su vez provocan nuevos atributos, uno de ellos es menor potencia de
cálculo en comparación con otro tipo de sistemas. Además, en función de las
condiciones ambientales donde se encuentre el SE es posible que necesite ser
dotado de protección adicional ante situaciones especiales de temperatura,
humedad, vibración, etc.

Ciertos SE controlan el funcionamiento de otros sistemas que requieren que sus
operaciones se realicen de manera determinista. Es decir, una instrucción dada
se tiene que realizar de manera inmediata o con un retardo mínimo y conocido de
antemano. Para dar soporte a operaciones en tiempo real, los SE cuentan con
Sistemas Operativos en Tiempo Real (RTOS) que se encargan de repartir el tiempo
de ejecución de cada tarea, asignándolo en función de la prioridad de cada
tarea.

A nivel de \extranjerismo{hardware}, un microcontrolador es quien se encarga de
ejecutar las instrucciones programadas. Cada microcontrolador cuenta con un
microprocesador, con memoria y con determinados periféricos de entrada y salida.
Para que el SE se pueda comunicar con sus periféricos o con otros dispositivos
se emplean los buses serie de datos. Algunos de los buses más utilizados son
Inter-Integrated Circuit (I\textsuperscript{2}C), Serial Peripheral Interface
(SPI), o Universal Serial Bus (USB).

Otro medio que emplean los SE para comunicarse son las redes de comunicaciones
de datos a las que se puede conectar tanto por cable como de manera
inalámbrica. Existen varios protocolos para transmitir datos en red, un conjunto
muy común y conocido es la familia de protocolos de internet, también
conocido como pila o conjunto de protocolos TCP/IP
(Transmission Control Protocol / Internet Protocol). Esta familia de protocolos
es ampliamente utilizada para comunicar equipos en redes de área local (LAN), en
redes de área local inalámbricas (WLAN) y en todo el mundo a través de Internet.

Con el uso de TCP/IP se abre un abanico de nuevas funcionalidades permitiendo
realizar funciones, apoyadas en otros protocolos de la pila TCP/IP, que antes
no eran posibles. Por ejemplo, usando el Hypertext Transfer Protocol (HTTP) se
pueden transferir páginas web, con File Transfer Protocol (FTP) transferir
archivos, con Simple Mail Transfer Protocol (SMTP) enviar correos electrónicos
o con Message Queuing Telemetry Transport (MQTT) se pueden enviar
mensajes bajo el patrón de mensajería de publicar-suscribir.

En este trabajo se muestra como crear un SE conectado usando TCP/IP en una placa
de desarrollo FRDM-K64F del fabricante NXP. Se parte de la configuración de los
componentes \extranjerismo{hardware} y \extranjerismo{software} necesarios, para
terminar demostrando la interacción, de manera remota, con el SE desde una
aplicación web. Desde dicha aplicación es posible enviar comandos para realizar
alguna de las funciones programadas en la placa.

\section{Estructura de la memoria}\label{sec:estructura}
La presente memoria se estructura de la siguiente manera:

\begin{itemize}
\item
  \textbf{Introducción:} descripción abreviada de los temas principales 
  abordados en el proyecto. La introducción está acompañada de la estructura
  que toma la memoria y un listado con el contenido adjunto a la misma.
\item
  \textbf{Objetivos del proyecto:} declaración de los objetivos generales,
  técnicos y personales que se pretenden conseguir con el desarrollo de este
  trabajo.
\item
  \textbf{Conceptos teóricos:} explicación de los conceptos teóricos más
  relevantes en la realización del proyecto. Se tratan tanto los sistemas
  empotrados como la familia de protocolos TCP/IP.
\item
  \textbf{Técnicas y herramientas:} descripción de las técnicas y las
  herramientas que han sido empleadas para el desarrollo del proyecto. También
  se muestran las alternativas valoradas y los motivos de escoger la opción
  seleccionada.
\item
  \textbf{Aspectos relevantes del desarrollo:} presentación de los aspectos o
  facetas que han tomado mayor relevancia durante la ejecución de trabajo.
\item
  \textbf{Trabajos relacionados:} estado de la técnica en la creación de
  sistemas que emplean sistemas empotrados conectados en red.
\item
  \textbf{Conclusiones y líneas de trabajo futuras:} conclusiones extraídas tras
  la realización del proyecto, así como nuevas líneas de trabajo sobre las que
  mejorar o ampliar lo presentando en este proyecto.
\end{itemize}

\section{Anexos a la memoria}\label{sec:anexos}
La memoria se presenta acompañada de los siguientes anexos:

\begin{itemize}
\item
  \textbf{Plan del proyecto software:} exposición de la
  planificación temporal y del estudio de viabilidad del proyecto, tanto la
  económica como la legal.
\item
  \textbf{Especificación de requisitos del software:}
  presentación del catálogo de requisitos, así como la descripción de cada uno
  de ellos.
\item
  \textbf{Especificación de diseño:} descripción de la fase de diseño, el
  diseño de datos, el diseño procedimental y el diseño arquitectónico, del
  \extranjerismo{software} del SE y de la aplicación web.
\item
  \textbf{Manual del programador:} explicación en detalle de aquellos aspectos
  que un programador debe conocer para trabajar con el código fuente del
  proyecto.
\item
  \textbf{Manual de usuario:} explicación para que un usuario interesado en el
  proyecto sea capaz de instalar, configurar y operar con el SE y con la
  aplicación web.
\end{itemize}

\clearpage

\section{Contenido adjunto}\label{sec:adjunto}
Se adjunta el siguiente contenido a la memoria y los anexos:

\begin{itemize}
\item
  \extranjerismo{Software} para la placa de desarrollo \cite{webpage:repo-se}.
\item
  Aplicación web para interacción remota \cite{webpage:repo-aw}.
\item	
  Documentación del \extranjerismo{software} \cite{webpage:repo-se-doc}.
\item	
  Documentación de la aplicación web \cite{webpage:repo-aw-doc}.
\item
  Repositorio en línea con el código del \extranjerismo{software} \cite{webpage:repo-se}.
\item
  Repositorio en línea con el código de la aplicación web \cite{webpage:repo-aw}.
\end{itemize}

\capitulo{2}{Objetivos del proyecto}\label{ch:objetivos}

En este capítulo se detallan los objetivos que se pretenden conseguir con la
ejecución del proyecto. Se diferencian tres tipos de objetivos, los generales
que dan causa al proyecto, los técnicos inherentes al tipo de proyecto realizado
y los personales que se desean conseguir \extranjerismo{motu proprio}.

\section{Objetivos generales}\label{sec:obj_generales}
\begin{itemize}
  \item Configurar un sistema empotrado que sea capaz de conectarse en red.
  \item Dotar al sistema empotrado de diversas funciones usando algunos de los
  periféricos de los que dispone.
  \item Demostrar la ejecución correcta de las funciones implementadas.
  \item Crear una interfaz web que permita interactuar con el sistema empotrado.
\end{itemize}

\section{Objetivos técnicos}\label{sec:obj_tecnicos}
\begin{itemize}
  \item Configurar el \extranjerismo{hardware} de una placa de desarrollo
  FRDM-K64F para poder conectarla a través de su puerto Ethernet a una LAN.
  \item Utilizar componentes \extranjerismo{hardware} de la placa como 
  son sus LED de colores para mostrar que es capaz de comunicarse via TCP/IP.
  \item Utilizar el bus serie de datos I\textsuperscript{2}C presente en la
  placa para extender su funcionalidad y que pueda mostrar mensajes en una
  pantalla.
  \item Emplear modulación por ancho de pulsos para regular la intensidad del
  brillo de unos LED presentes en una placa de expansión.
  \item A nivel de comunicaciones, usar la implementación ligera de 
  TCP/IP ``lwIP'' para manejar las transmisiones.
  \item Crear una aplicación web usando la tecnología JSF capaz de comunicarse
  con una placa conectada en red.
\end{itemize}

\section{Objetivos personales}\label{sec:obj_personales}
\begin{itemize}
  \item Ampliar los conocimientos y la experiencia en el desarrollo de
  \extranjerismo{software} para sistemas empotrados.
  \item Conocer el conjunto de herramientas de trabajo que proporciona el
  fabricante de la placa de desarrollo.
  \item Extender el entendimiento de la familia de protocolos TCP/IP.
  \item Revisar el proceso de desarrollo de una aplicación web.
  \item Emplear el sistema de control de versiones distribuido Git a través
  de la plataforma de desarrollo GitHub.
  \item Emplear la técnica de desarrollo ágil Scrum en las diferentes fases del
  desarrollo.
  \item Aprender a usar el sistema de composición de textos \LaTeX\ y utilizarlo
  para realizar la documentación del proyecto.
  \item Aplicar las competencias adquiridas a lo largo de las diferentes
  asignaturas que componen el Máster Universitario en Ingeniería Informática.
\end{itemize}

\capitulo{3}{Conceptos teóricos}
En este capítulo se sintetizan algunos de los aspectos tratados en este proyecto
para mejorar su compresión y entendimiento.

\section{Sistemas empotrados}
Parte importante del proyecto se centra en obtener un sistema empotrado capaz de
comunicarse usando los protocolos TCP/IP. A continuación se describen los
conceptos más relevantes en torno a los SE.

\subsection{Descripción}
Se puede considerar que un sistema empotrado es aquel cuyo
\extranjerismo{hardware} y \extranjerismo{software} se encuentran estrechamente
relacionados, está diseñado para cumplir con una función específica, se haya
integrado en un sistema mayor, no se espera que el usuario lo modifique y puede
trabajar sin interacción o con la mínima interacción humana necesaria.
\cite{jime13}

\imagen{refrigerator}{Ejemplo de integración de SE. \cite{nxp01}}

En la figura \ref{fig:refrigerator} se muestra un ejemplo de uso de varios SE
dentro de un sistema mayor, en este caso un frigorífico inteligente. El sistema
cuenta con varios componentes, la interfaz de usuario, la gestión del sistema,
el control del motor o la conectividad. Cada uno de los componentes se ayuda de
un SE para realizar su función asignada.

Los SE cuentan con el \extranjerismo{hardware} específico para la tarea a
realizar. La interfaz de usuario puede contar con pantallas o botones. La
gestión del sistema tiene acceso a sensores, el control de la corriente, la
iluminación o los ventiladores. El control del motor presenta componentes
eléctricos para la regulación del compresor. Un módulo de energía, un triodo
para corriente alterna o un módulo de corrección del factor de potencia son
componentes que pueden estar presentes en el control del motor. Asimismo, el
\extranjerismo{software} ejecutado en cada uno de estos SE varía según la
función a desempeñar realizando únicamente las tareas necesarias.

Por otra parte, también se puede advertir que los SE no están pensados para que
el usuario los modifique o programe, ni requieren que la interacción sea
constante por parte del usuario para su correcto funcionamiento.

Los SE y los sistemas que los emplean se encuentran fácilmente. Se hallan en 
sistemas de movilidad y transporte, automatización industrial, sector sanitario,
edificios inteligentes, redes de suministro inteligentes, investigación
científica, seguridad pública, supervisión de salud estructural, recuperación de
desastres, robótica, agricultura y ganadería, aplicaciones militares,
telecomunicaciones y electrónica de consumo. \cite{marw18}

\subsection{Características del \extranjerismo{hardware}}
Los SE disponen de componentes \extranjerismo{hardware} que siendo específicos
para la tarea a la que están destinados se pueden generalizar en procesador,
memoria y puertos de entrada y salida. El procesador se encarga de ejecutar las
instrucciones de los programas que manejan las entradas y las salidas del
sistema. Los programas ejecutados y los datos generados se almacenan en la
memoria. Y los puertos de entrada y salida, envían y reciben la señales con las
que trabaja el procesador.\cite{jime13}

También existen otros elementos que se encuentran a menudo en los SE:
\begin{itemize}
    \item Puertos de comunicación serie o paralelo
    \item Dispositivos de interfaz humana
    \item Sensores
    \item Actuadores
    \item Conversores analógica-digital (ADC)
    \item Conversores digital-analógica (DAC)
    \item Componentes de diagnóstico y redundancia
    \item Componentes de apoyo al sistema
    \item Otros subsistemas:
    \begin{itemize}
        \item Circuito integrado para aplicaciones específicas (ASIC)
        \item Matriz de puertas programables (FPGA)
    \end{itemize}
\end{itemize}

En la figura \ref{fig:diagr_bloques_hw} se presenta un diagrama de bloques
en el que se puede observar de forma general los componentes que forman un SE
y su funcionamiento. Las entradas son procesadas por el microcontrolador que a
su vez generará las salidas apropiadas. 
\imagen{diagr_bloques_hw}{Diagrama de bloques del HW de un SE}

\subsection{Características del \extranjerismo{software}}
En cuanto al \extranjerismo{software} presente en los sistemas embebidos, se
pueden diferenciar varios grupos. El primero de ellos los
\extranjerismo{drivers} o controladores encargados de la interacción directa
con el \extranjerismo{hardware} del SE.

Luego se encuentra el \extranjerismo{middleware}, que es aquel
\extranjerismo{software} que ocupa una posición entre el sistema operativo y los
programas. El término se usa con frecuencia en librerías de rutinas de
infraestructura que proporcionan servicios a los desarrolladores de los
programas. \cite{butt16}

Junto a lo anterior se puede hallar un Sistema operativo en tiempo real (RTOS).
En un sistema operativo de propósito general varias tareas se ejecutan de forma
aparentemente simultánea. De este modo se puede repartir el tiempo de ejecución
de manera equitativa entre usuarios, por ejemplo. En cambio, en un RTOS se prima
la ejecución de las tareas en un tiempo estrictamente limitado. Con un sistema
de prioridades se determina la importancia de las tareas y cuales necesitan ser
realizadas sin demora. \cite{amaz19}

Por último y funcionado sobre lo anterior se encuentra los programas necesarios
para el funcionamiento del SE. En caso de contar con un RTOS será este el que se
encargue de ejecutar una tarea u otra. Sino, de forma conocida como
\extranjerismo{bare-metal} se ejecutan las tareas de forma secuencial. Secuencia
que solo es alterada en caso surgir una interrupción.

\imagen{diagr_bloques_sw}{Digrama de bloques del SW de un SE}
En el diagrama de bloques de la figura \ref{fig:diagr_bloques_sw} se observan
como se ubican unos componentes sobre otros. Los componentes se comunican con
aquellos adyacentes. El RTOS es un componente opcional y que funciona de forma
paralela a \extranjerismo{drivers} y \extranjerismo{middleware}.

\capitulo{4}{Técnicas y herramientas}{\label{sec:tecyherra}}
Este capítulo presenta las técnicas metodológicas y las herramientas de
desarrollo usadas durante la realización del proyecto. En algunas de las
técnicas y herramientas existen varias alternativas a utilizar. En este capítulo
se comentan las características principales de cada alternativa y se expone el
porqué de la elección tomada.

\section{Técnicas metodológicas}{\label{sec:tecnicas}}

\subsubsection{Scrum}{\label{sec:scrum}}
Scrum es definido por dos de sus mayores promotores \textcite{schwaber2017}
como:
\begin{quotation}``un marco de trabajo en el que las personas pueden
  abordar problemas flexibles y complejos, mientras se entregan productos creativa y
  productivamente del mayor valor posible''
\end{quotation}.

Desde el punto de vista del desarrollo del \extranjerismo{software}, trabajar en
un producto en toda su extensión puede atraer problemas debidos, por ejemplo, a
cambios en los requisitos de los clientes. Para solventar este problema, Scrum
enuncia la realización del trabajo de forma ligera, iterativa e incremental,
permitiendo que el desarrollo pueda responder mejor ante imprevistos.

Scrum propone grupos de trabajo donde sus integrantes tienen diferentes roles y
responsabilidades. El grupo sigue un flujo de trabajo que gira entorno al
concepto de \extranjerismo{sprint}, definido como la unidad básica de
desarrollo. Antes de cada \extranjerismo{sprint} se planifica su duración
temporal, se definen la tareas a realizar en él y cuando ya está en marcha se
revisa diariamente para analizar su evolución. Una vez terminado se comprueban
los resultados obtenidos y se proporcionan los entregables generados.

\section{Herramientas \extranjerismo{hardware} de desarrollo}
  {\label{sec:herramientas}}

\subsubsection{Placa de desarrollo FRDM-K64F}{\label{sec:k64f}}
El uso de placas de desarrollo sirve como toma de contacto al microprocesador y
al resto del \extranjerismo{hardware} que componen un SE. En este proyecto la
placa utilizada es el modelo
\href{https://www.nxp.com/support/developer-resources/evaluation-and-development-boards/freedom-development-boards/mcu-boards/freedom-development-platform-for-kinetis-k64-k63-and-k24-mcus:FRDM-K64F}
{FRDM-K64F}.

\imagen{k64f}{Placa de desarrollo FRDM-K64F \cite{webpage:k64f}}

El microcontrolador (MCU) de esta placa es el \href{https://www.nxp.com/products/processors-and-microcontrollers/arm-based-processors-and-mcus/kinetis-cortex-m-mcus/k-seriesperformancem4/k6x-ethernet/kinetis-k64-120-mhz-256kb-sram-microcontrollers-mcus-based-on-arm-cortex-m4-core:K64_120}
{Kinetis K64}. Este MCU cuenta con el microprocesador \href{https://developer.arm.com/products/processors/cortex-m/cortex-m4}
{Cortex-M4} capaz de funcionar a 120 Mhz. Destaca la capacidad de funcionar en 
modos de energía ultra bajos, su prominente rendimiento y los dispositivos
integrados de conectividad y comunicación como los puertos General Purpose
Input/Output (GPIO), la interfaz Ethernet o los buses de datos serie
I\textsuperscript{2}C o SPI.

La placa K64F ofrece el \extranjerismo{hardware} necesario para poder aprovechar
las características que ofrece el MCU. Cuenta con los pines suficientes para dar 
cabida a los GPIO. La disposición de los conectores es compatible con la
utilizada en placas \href{https://www.arduino.cc/}{Arduino} permitiendo el uso
de multitud de placas de extensión ya existentes.

También presenta un puerto Ethernet que permite conectar la placa a una red
local y, a la postre, a Internet. También cuenta con \extranjerismo{hardware}
extra que posibilita nuevas opciones a la hora de investigar y desarrollar con
la placa. Por ejemplo, puertos USB, para alimentación, depuración y conexión, y
diodos emisor de luz (LED) de colores rojo, verde y azul (RGB).

\subsubsection{Placa de expansión Arduino Basic I/O}{\label{sec:basic-io}}
Como la placa K64F es compatible con las placas de expansión de Arduino se
utiliza la placa \href{https://web.archive.org/web/20160818213905/http://www.msebilbao.com/tienda/product_info.php?cPath=130&products_id=793&osCsid=f967e6ddeaaa2f19050972ff62295a08}
{Arduino Basic I/O} para poder utilizar \extranjerismo{hardware} adicional.

\imagen{basic_io}{Placa de expansión Arduino Basic I/O \cite{webpage:basio-io}}

En concreto, se utilizan sus 4 LED de colores que usados en combinación de la
técnica de modulación de señales pulse-width modulation (PWM) se puede regular
la intensidad del brillo de cada uno de los LED.

\subsubsection{Pantalla de cristal líquido (LCD)}{\label{sec:lcd}}
El disponer de una pantalla permite mostrar información en forma de texto a un
usuario del SE. En particular, esta pantalla puede mostrar 2 líneas de texto
de 16 caracteres cada una. Gracias a esta pantalla el usuario puede
conocer ciertos datos acerca del estado de inicialización de la placa, o puede
usarla para mostrar sus propios mensajes.

Cabe destacar que la pantalla usa una conexión de tipo paralela, necesitando
hasta 16 pines para la transmisión de datos, la alimentación, la iluminación y
el control de la transmisión. Como forma de simplificar su uso la pantalla
incorpora un módulo de comunicación I\textsuperscript{2}C, de tal manera que
solo es necesario el empleo de 4 pines.

\subsubsection{Accesorios extra}{\label{sec:extras}}
Para realizar el montaje de las placas y la pantalla se utilizan varios
accesorios extra.

Las placas de pruebas \cite{webpage:placa-pruebas} son unos tableros con
orificios que permiten conectar otras placas, componentes electrónicos,
conectores y cables. Su orificios están conectados eléctricamente, siguiendo un
patrón de bloques y líneas, permitiendo así la conexión entre los elementos del
sistema a montar. La rapidez y facilidad del montaje y desmontaje de los
elementos facilita el prototipado de sistemas antes de su fabricación.

Con el fin de conectar los pines de los componentes se usan varios cables
puente \cite{webpage:cable-puente}. Algunos conectados directamente a las
placas, otros conectados a las placas de pruebas. Para evitar confusiones los
cables son de colores, de tal manera que es posible distinguir la función que
realiza cada uno.

Por último, como el objetivo del proyecto es realizar un SE que utilice los
protocolos TCP/IP, es necesario conectar la placa K64F con un cable Ethernet
a un determinado equipamiento de acceso a la red, por ejemplo,
un \extranjerismo{switch}.

\capitulo{5}{Aspectos relevantes del desarrollo del proyecto}
{\label{ch:aspectos}}
Durante el desarrollo del proyecto han surgido diversos hechos relevantes. En
este capítulo se comentan los eventos ocurridos y la decisiones tomadas al
respecto.

\section{Aprendizaje}{\label{sec:aprendizaje}}
Una faceta fundamental del proyecto ha sido el aprendizaje. Conocer nuevas 
herramientas, técnicas y metodologías, profundizar en las ya conocidas y revisar
aquellas que hubieran podido caer en el olvido.

Revisando los objetivos personales \ref{sec:obj_personales} se puede ver como
el deseo de aprendizaje está presente al dejar patente que no solo se quiere
cumplir con los objetivos generales \ref{sec:obj_generales} y técnicos 
\ref{sec:obj_tecnicos}.

Querer utilizar nuevas herramientas, técnicas, metodologías, etc. no está exento
de problemas o contratiempos que no dudaron en dejarse ver desde el inicio.

\section{Inicio}{\label{sec:inicio}}
Para desarrollar el \extranjerismo{software} que va a utilizar la placa K64F
es necesario usar un IDE. En la sección sobre los IDE valorados \ref{sec:ide}
se comentan brevemente las diferencias entre Kinetis Design Studio (KDS) y
MCUXpresso.

La decisión de usar MCUXpresso estuvo basada fundamentalmente en el abandono
por parte de NXP de KDS. Esto no quiere decir que no se pudiera haber usado KDS.
Pero siguiendo la propia recomendación del fabricante y queriendo usar
herramientas modernas se tomó la decisión de usar MCUXpresso.

La principal diferencia entre IDE es la presencia de Processor Expert (PE) en
KDS y la respectiva ausencia en MCUXpresso. Esta herramienta fue la usada en su
momento para aprender a desarrollar SE. Al perder PE fue necesario dedicar
tiempo a aprender como usar las nuevas herramientas presentes en la
\extranjerismo{suite} de MCUXpresso, el IDE, el SDK y las Config Tools.

La primera toma de contacto estuvo apoyada en los ejemplos ofrecidos por el SDK.
Son pequeñas piezas de código que realizan funciones simples que permiten
experimentar con la placa y añadir modificaciones para ver como responde.

Familiarizado con el IDE, tocaba aprender a configurar los relojes del sistema.
Usar como base la configuración de los ejemplos sirvió para conocer que relojes
se tenían que activar, a que frecuencia y que salidas se tenían que habilitar.

Por último queda configurar el MCU y sus pines. Para poder usar un componente
hay que decir al MCU que pines se van a utilizar y la función deseada. Que un
dispositivo funcione correctamente requiere de la sección de los pines 
correctos y de la configuración eléctrica correcta. Por ejemplo,
para poder usar el puerto Ethernet no sirve la configuración por defecto sino
que hay que configurar el estado \extranjerismo{pull-up} de alguno de sus pines.

Dominado MCUXpresso se pudo comenzar el desarrollo propiamente dicho.

\section{Desarrollo con la placa K64F}{\label{sec:desarrollo-k64f}}

\subsection{Técnicas y metodologías}{\label{sec:desarrollo-tym}}
En esta fase se tomó la decisión junto con el tutor, de integrar metodologías y 
técnicas que podían resulta beneficiosas para el proyecto.

Una medida tomada fue usar GitHub como repositorio del código y sistema de 
control de versiones. Además, como los IDE y VS Code están preparados para
usar Git, su uso resultó bastante asequible y cómodo.

Otra medida tomada fue usar Scrum y para ello el trabajo se planificó en
sucesivos \extranjerismo{sprints}. Scrum propone equipos de desarrollo
de varias personas, reuniones diarias, reuniones al terminar el
\extranjerismo{sprint}, el rol de Scrum Master, el Product Owner, etc.
Como el proyecto se ha realizado individualmente, usar Scrum ha servido como
forma de conocer la metodología, usando principalmente el \extranjerismo{sprint}
como herramienta para dividir el trabajo y planificar el proyecto.

\subsection{Conexión a la red}{\label{sec:desarrollo-red}}
Con la placa siendo capaz de usar el puerto Ethernet se presentaba el asunto
de la obtención del dirección IP. Establecer una dirección fija presenta
problemas si ya está usada o si se conecta en una subred con diferente rango
de direcciones.

Por este motivo se tomó la decisión de usar DHCP. Esta decisión
obliga a contar con un servidor DHCP en la misma red a la que se conecta la
placa. De no obtener dirección la placa se quedaría a la espera indefinidamente.
Para que el usuario sepa la dirección de la placa, en un primer momento se
mostraba en la consola de depuración para más tarde mostrar también en el LCD
tanto la dirección IP como el puerto TCP a la escucha.

Con la pila TCP/IP en funcionamiento era posible enviar paquetes de datos a la
placa. Como lwIP implementa varios protocolos, ICMP entre ellos, era posible
realizar \extranjerismo{pings} para comprobar el estado de la placa.

\subsection{Funciones del \extranjerismo{hardware}}{\label{sec:desarrollo-hw}}
Para ejemplificar el uso de un SE conectado en red se decidió aprovechar el
\extranjerismo{hardware} incluido en la placa. Representando a unos actuadores y 
siendo los componentes más visibles de la placa se decidió utilizar los LED de
colores RGB. Para poder activar o desactivar los LED se determinó enviar un
comando con una instrucción y argumento específicos para el LED a alterar. El
comando se transmitiría dentro un paquete TCP.

El parámetro de cada comando sirve para indicar el color a alterar.
Los comandos para encender los tres colores básicos de los LED RGB son:
\begin{quotation}
  ``led:r''

  ``led:g''

  ``led:b''
\end{quotation}

Como los LED se pueden encender simultáneamente se puede combinar su luz para
crear nuevos colores. En la tabla siguiente se pueden ver los todos 
colores posibles. 

\tablaSmall{Colores producibles usando los LED RGB}
{l c c c}{coloresrgb}
{\multicolumn{1}{l}{Color} & LED Rojo & LED Verde & LED Azul\\}
{
  Rojo     & X &   &   \\
  Verde    &   & X &   \\
  Azul     &   &   & X \\
  Amarillo & X & X &   \\
  Magenta  & X &   & X \\
  Cyan     &   & X & X \\
  Blanco   & X & X & X \\
  Ninguno  &   &   &   \\
}

El LCD se incluyó con la intención de mejorar la comunicación del usuario y
ampliar la funcionalidad del SE. Esta decisión provocó inconveniente que volvía
a involucrar el uso de MCUXPresso. Con anterioridad para poder utilizar el LCD
se usaba una librería de funciones adaptada para usar los componentes generados
por PE. 

En origen la librería ya había sido adaptada desde código para Arduino a
código para PE. Así que surgió la necesidad de adaptar de nuevo la librería para
usar el código generado por MCUXpresso. Como la librería usaba funciones de
espera también se tuvieron que portar estas funciones.

El comando a transmitir indica que la operación se trata de un mensaje para la
pantalla, la línea en la que mostrar el mensaje y la cadena de caracteres a
presentar.
Ejemplos de comandos para mostrar una cadena de caracteres en cada línea:
\begin{quotation}
  ``msg:0:Lorem ipsum''

  ``msg:1:dolor sit amet''
\end{quotation}

La última función añadida fue modificar la intensidad del brillo de los LED
incorporados en la placa de expansión. La regulación de la intensidad se
consigue usando los pines compatibles con PWM del MCU. Los pines PWM no
están comunicados con los LED RGB de la propia placa K64F así que se conectaron
a los pines de los LED de placa de expansión.

Al transmitir un comando y tras indicar que es una instrucción de tipo PWM,
los argumentos indican el LED a regular y la intensidad (de 0\% a 100\%)
deseada. Por ejemplo, los siguientes comandos sirven para regular la intensidad
de cada LED con incrementos del 25\%.
\begin{quotation}
  ``pwm:w:0''

  ``pwm:g:25''

  ``pwm:y:50''

  ``pwm:r:100''
\end{quotation}

\subsection{Montaje \extranjerismo{hardware}}{\label{sec:desarrollo-montaje}}
Una vez que cada componente funcionaba de la manera esperada se realizó
el montaje en las placas de pruebas para poder hacer funcionar el SE al
completo. 

\subsection{Tareas del RTOS}{\label{sec:desarrollo-rtos}}
Como se ha visto en la sección sobre \extranjerismo{software} de los SE 
\ref{sec:se-sw}, es habitual contar con un RTOS encargado de la planificación
de las tareas ejecutadas. El RTOS a usar fue FreeRTOS \ref{sec:otros} por su
conocimiento y experiencia previa. Para cada función disponible en la placa se
creó una tarea concreta. Es decir, hay una tarea para usar el LCD, otra para los
LED de la placa de expansión y otra para los LED RGB de la propia placa. 
Las tareas quedan a la espera de recibir un comando de la tarea, de mayor
prioridad, que está escuchando en la dirección y el puerto asignados.

\subsection{Desarrollo de la aplicación web}{\label{sec:desarrollo-app}}
Con la placa operativa llegaba el turno de crear una aplicación web que
permitiese actuar remotamente con la placa. La tecnología para crear la
aplicación web es JSF, la parte encargada de comunicarse con la placa escrita
en Java y la página visible usada por el usuario en XHTML y CSS.

Al desconocer CSS se presentaba una oportunidad de aprender sobre ese lenguaje.
De nuevo requería dedicar cierto tiempo al estudio y aprendizaje hasta lograr
una página con un aspecto aceptable.

Al usuario se le muestran cuatro apartados. El primero le permite introducir la
dirección IP y el puerto TCP mostrados por el LCD. En caso de existir varios SE
se pueden cambiar estos parámetros para escoger el SE con el que interactuar.

El siguiente bloque muestra un cuadrado por cada uno de los colores que se
pueden generar con los LED RGB. Si lo que se desea es apagar los LED, pulsando
el cuadrado negro se apagan todos.

El tercer apartado permite introducir una cadena de texto que será enviada al
LCD. Como la pantalla solo puede mostrar 16 caracteres los cuadros de texto
están limitados a esta longitud. Se muestran dos cuadros, uno para cada línea
del LCD.

El último bloque está compuesto por cuatro controles deslizantes. Hay uno por
LED de la placa de expansión. Usando estos controles se puede ajustar de forma
visual la intensidad, yendo de 0\% a 100\% deslizando el control de izquierda
a derecha.

Como todo el contenido se muestra en una página era necesario modificar el
comportamiento realizado tras pulsar los botones. Por defecto, al pulsar un
botón la página se desplazaba hasta arriba. El inconveniente se solucionó
inyectando código HTML en los botones para modificar su comportamiento.

\section{Documentación}{\label{sec:documentacion}}
En cuanto a la documentación, más allá de aprender a utilizar \LaTeX{} no se
presentaron mayores inconvenientes. Aparecieron las dudas habituales que surgen
cuando se está aprendiendo a usar nuevas herramientas o técnicas pero se fueron
solventando a medida que fueron surgiendo.

El uso de la plantilla facilitada por el tutor allanó la creación y edición de 
los documentos de la memoria y anexos. Se realizaron pequeñas modificaciones a
la plantilla con la intención de mejorar el resultado obtenido. 

Uno de los cambios fue usar el paquete \emph{biblatex-iso690} para generar la
bibliografía de acuerdo a la norma UNE-ISO 690:2013. Otra adición fue el
paquete \emph{url} que se encarga de mejorar la forma en que se tratan las URL
en el texto.

\capitulo{6}{Trabajos relacionados}{\label{ch:relacionados}}
En el campo de los SE se han realizado sistemas complejos y de gran valor en 
numerosos ámbitos. En este sentido, \href{https://www.nxp.com/products/processors-and-microcontrollers/arm-based-processors-and-mcus/kinetis-cortex-m-mcus/k-seriesperformancem4/k6x-ethernet/kinetis-k64-120-mhz-256kb-sram-microcontrollers-mcus-based-on-arm-cortex-m4-core:K64_120}
{NXP} muestra varias aplicaciones de los SE en las que se pueden emplear el
mismo MCU, Kinetis~\textsuperscript{\tiny\textregistered} K64, existente en la
placa de desarrollo.

Los sistemas expuestos muestran características similares a las presentes en la
placa K64F y a las usadas en este proyecto. El uso de conectividad en red via
TCP/IP hace posible la comunicación con el SE a distancia. Y la utilización de
GPIO, I\textsuperscript{2}C o PWM permite operar con los sensores, actuadores u
otros componentes del SE.

A continuación se muestran cuatro sistemas pertenecientes a diferentes sectores
de aplicación. Los ejemplos pertenecen al sector industrial, sanitario y
comercial. En concreto, los sistemas operados por SE son los siguientes:

\begin{itemize}
  \item Vehículo no tripulado
  \item Sistema de telesalud
  \item Cama de hospital eléctrica
  \item Impresora de terminal punto de venta
\end{itemize}

La funcionalidad, desempeño y ámbito de uso de las aplicaciones resultan muy
diversas. Así pues, queda patente la versatilidad y el amplio espectro de
entornos en los que se pueden encontrar SE.

\subsection{Vehículo no tripulado}{\label{sec:uv}}
Un vehículo no tripulado es aquel vehículo capaz de funcionar sin un humano a
bordo. De forma general los vehículos no tripulados se clasifican en función del
medio en el que operan. Pueden ser terrestres, aéreos, acuáticos, submarinos,
espaciales... Independientemente del tipo que sean y de que estén operados por
control remoto o sean autónomos, los vehículos se sirven de SE para controlar
sus funciones.

\imagen{uv}{Diagrama de bloques de un vehículo no tripulado \cite{webpage:uv}}

Con LED controlados por GPIO se puede conocer el estado del vehículo. Para
dotar de conectividad al vehículo están disponibles interfaces cableadas e
inalámbricas.

El uso de PWM es una característica compartida con el proyecto. Mientras que 
en el proyecto se ha usado para regular la intensidad del brillo de unas
luces, en los vehículos no tripulados se emplea PWM para regular sus motores

Aunque no se ha incluido en el proyecto, la capacidad de usar tarjetas de
memoria, de dispositivos conectados a un puerto USB o la comunicación con
periféricos usando SPI está presente tanto en vehículos como en la placa de
desarrollo.

\subsection{Sistema de telesalud}{\label{sec:salud}}
El acceso a la asistencia sanitaria no siempre es factible o asequible. Vivir
en entornos rurales alejados de centros médicos, la falta de medios de
transporte, la pérdida de movilidad de los pacientes, o la falta de recursos
económicos son causas que pueden impedir dicha asistencia. Con el avance de la
tecnología y las telecomunicaciones ha surgido la posibilidad de abordar dicha
problemática con el uso de sistemas de telesalud.

Con el uso de sensores de diverso tipo se recogen datos sobre el estado de salud
del paciente. Su presión arterial, su ritmo cardíaco, su temperatura corporal,
etc. Los datos son recogido por el sistema, procesados y retransmitidos al
personal médico encargado de velar por la salud del paciente.

En la figura \ref{fig:telesalud} se muestran los componentes que integran el
sistema. Entorno al MCU se hallan presentes una interfaz infrarroja que recoge
los datos de los sensores y una pantalla que muestra información al paciente.
También se puede ver como el sistema usa PWM para generar sonidos.

Periódicamente el sistema envía los datos bien usando una conexión Ethernet o
bien usando una conexión USB a un ordenador conectado a Internet. De manera 
opcional se puede conectar el sistema a Internet de forma inalámbrica. Este
hecho serviría para aumentar la portabilidad del sistema de telesalud al
completo.

\imagen{telesalud}{Diagrama de bloques de un sistema de telesalud
\cite{webpage:telehealth}}

\subsection{Cama de hospital eléctrica}{\label{sec:cama}}
Siguiendo en el sector sanitario, se presenta otra muestra de aplicación de SE
en las camas de hospital eléctricas. El uso de SE permite ofrecer el 
máximo confort posible al paciente y mejorar atención sanitaria recibida.

Con una serie de elementos de regulación se puede acomodar la cama a las
necesidades del usuario. Para conseguir tal comodidad el SE cuenta con motores
encargados de regular la inclinación o altura de varias partes de la cama. 

Al integrar un monitor de constantes vitales o una bomba de infusión, 
dispostivos también creados con SE, se puede realizar una monitorización y 
atención completa del paciente. Y si además se conecta la cama a una red
inalámbrica se abre la posibilidad de monitorizar y atender remotamente al
paciente. Por ejemplo, desde un control de enfermería.

Para comunicarse telefónicamente con el enfermo se utiliza la conexión
Ethernet en conjunción con la tecnología voz sobre protocolo de internet (VoIP).

Por último, que la cama disponga de una pantalla LCD permite que conocer 
\extranjerismo{in situ} el estado de todo el sistema, paciente incluido.

\imagen{cama}{Diagrama de bloques de una cama de hospital eléctrica
\cite{webpage:bed}}

\subsection{Impresora de terminal punto de venta (TPV)}{\label{sec:tpv}}
Para terminar, se pasa al ámbito del sector minorista. Se muestra el sistema
utilizado en una impresora de un TPV. La impresora está compuesta por
dos SE. Mientras que uno de ellos se encarga de las entradas digitales, el
otro se ocupa del control de los motores y dispositivos de impresión.

Las opciones de conectividad permiten comunicarse con la impresora de varias
maneras. Con la conexión Ethernet se puede trabajar en red con la impresora.
Igualmente, se pueden usar las conexión USB o RS-485.

Para controlar el \extranjerismo{hardware} relativo al proceso de impresión
se cuenta con la ayuda de un segundo SE. Como el SE auxiliar recibe señales
analógicas requiere usar ADC, el resto de entradas y salidas se conectan
directamente a los pines GPIO. Finalmente, la comunicación con el SE principal
se realiza mediante I\textsuperscript{2}C o SPI.

Por otra parte, el SE principal cuenta con mayor número de conexiones al
recibir las entradas del usuario, los controles digitales, gestionar la
conectividad y manejar los sensores.

De nuevo están presentes características utilizadas a su vez en el proyecto:
conectividad en red a través de un puerto Ethernet, conexión con periféricos
usando el bus de datos serie I\textsuperscript{2}C y comunicaciones diversas via
GPIO.

\imagen{tpv}{Diagrama de bloques de una impresora de TPV\cite{webpage:tpv}}

\capitulo{7}{Conclusiones y Líneas de trabajo futuras}
{\label{ch:conclusiones-lineas}}

\subsection{Conclusiones}{\label{sec:conclusiones}}
Tras finalizar el proyecto han surgido varias conclusiones relativas al mismo.

\begin{itemize}
  \item Al concebir el proyecto se establecían una serie de objetivos generales
  \ref{sec:obj_generales}, técnicos \ref{sec:obj_tecnicos} y personales
  \ref{sec:obj_personales} que se han completado satisfactoriamente.
  Se ha logrado crear un SE \ref{sec:se} conectado en red gracias al uso de
  la familia de protocolos TCP/IP \ref{sec:tcpip}. De manera más específica,
  se ha utilizado una placa de desarrollo K64F \ref{sec:k64f} junto con lwIP,
  una implementación de la pila de protocolos TCP/IP \ref{sec:otros} para
  conectar en red dicha placa.
  
  \item Para su realización, se han puesto en práctica los conocimientos 
  aprendidos en el trascurso del Máster Universitario en Ingeniería Informática.
  Conocimientos en sistemas empotrados, arquitecturas y servicios de internet,
  gestión y dirección de proyectos, etc.

  \item A la vez se ha aprovechado esta oportunidad para adquirir nuevos
  conocimientos. Como se indica en los aspectos relevantes del proyecto
  \ref{sec:aprendizaje}, uno de los principios esenciales del proyecto ha sido
  aprender nuevas herramientas, técnicas y metodologías; y revisar aquellas ya
  conocidas.

  \item Ha resultado un descubrimiento conocer la cantidad de contextos en los
  que puede aparecer un SE. Era conocida su presencia en entornos con fuerte
  presencia de la tecnología de la información, p. ej., electrónica de consumo,
  automatización industrial, control de viviendas y edificios, interfaces
  hombre-máquina;pero su aparición en otros ámbitos como el sector de la salud
  donde su uso se extiende desde camas eléctricas \ref{sec:cama} hasta bombas
  de insulina ha sido sorprendente.

  \item Desarrollar un SE se convierte en un desafío en comparación con el
  desarrollo para sistemas convencionales. Las restricciones del 
  \extranjerismo{hardware} se traducen en una necesidad constante de ahorrar
  espacio en memoria, limita la cantidad y el tamaño de librerías a usar,
  insta a usar implementaciones livianas como en el caso de lwIP y dado el
  caso que lo requiera, apremia el uso de soluciones que reduzcan el consumo
  de energía por parte del sistema.

  \item Aunque la pila de protocolos TCP/IP no sean los únicos tipos de 
  protocolos usados por los SE su empleo ha facilitado el desarrollo con la
  placa y su posterior uso. Con inicializar la pila de protocolos era suficiente
  para poder enviar un \extranjerismo{ping} y ver como la  placa respondía 
  satisfactoriamente.

  \item Los mecanismos de depuración incluidos en la placa de desarrollo y en el
  IDE \ref{sec:ide} han demostrado su valía. En numerosas ocasiones mostrar
  en la consola de depuración un mensaje con el estado de la ejecución ha sido
  de gran utilidad para corregir errores. Sin ir más lejos, hasta la
  incorporación de la pantalla LCD, un mensaje de depuración era la única manera
  de saber la dirección IP asignada a la placa.

  \item Por lo que se refiere al desarrollo de la aplicación web, el aprendizaje
  y empleo de CSS ha servido para mostrar que se puede hacer en términos de
  presentación y diseño visual; y como se puede hacer. El desarrollo también
  ha servido para revisar las tecnologías usadas: Java EE, Enterprise Beans,
  Java Server Faces, GlassFish, Maven, etc.
 
  \item Con el uso de Scrum se consiguió repartir las tareas del desarrollo
  de forma práctica y efectiva. Siguiendo la planificación general, el
  desarrollo se efectuó en función del \extranjerismo{sprint} del momento,
  concentrando los esfuerzos en la meta fijada en él.

  \item Por último, citar el empleo de Git y de GitHub \ref{sec:vcs}. Su
  utilización ha evidenciado de nuevo su utilidad e idoneidad. En más de una
  ocasión ha sido valiosa la capacidad de poder recuperar el código fuente de
  una versión estable. Además, al estar disponible en línea ha permitido al
  tutor revisar y orientar sobre la evolución del desarrollo.
\end{itemize}

\subsection{Líneas de trabajo futuras}{\label{sec:lines-futuras}}
El proyecto realizado puede ser mejorado o ampliado en diversos frentes.
A continuación se describen aquellos aspectos de mayor relevancia.

\begin{itemize}
  \item A nivel de \extranjerismo{hardware} la placa de desarrollo tiene
  integrados más dispositivos de los utilizados en el proyecto con los que se
  podría tratar. Los botones físicos, el acelerómetro y magnetómetro, los
  canales de comunicación UART, SPI, USB, ADC; o el lector de tarjetas de
  memoria podrían proporcionar nuevas funcionalidades.
  
  \item En línea con lo anterior, la placa cuenta con la posibilidad de añadir
  módulos de conexión inalámbrica Wi-Fi y Bluetooth, de esta manera, resultaría
  interesante crear un SE conectado en red y móvil.
  
  \item El concepto de internet de las cosas (IoT) cada vez está más presente y
  la cantidad de dispositivos inteligentes conectados en red está aumentando
  rápidamente. Una mejora del proyecto vendría de la mano de la
  integración en la pila TCP/IP de alguno de los protocolos usados en IoT. Por
  ejemplo, MQTT o CoAP.

  \item En ese sentido, usando protocolos IoT se podría conectar el SE a la
  nube. Servicios como \href{https://azure.microsoft.com/es-es/services/iot-hub/}
  {Azure IoT Hub} permiten conectar, supervisar y administrar multitud de
  dispositivos inteligentes.

  \item Con esa orientación hacia IoT, también existen RTOS dedicado al asunto.
  \href{https://www.zephyrproject.org/}{Zephyr} es compatible con la placa K64F
  y además integra opciones de seguridad y protección no siempre presentes en
  SE actuales.

  \item También es cierto que hay medidas de seguridad que se pueden integrar
  directamente. Aprovechando que lwIP implementa el protocolo criptográfico
  Transport Layer Security (TLS) las comunicaciones podrían ser cifradas para 
  proporcionar seguridad en la transmisión de datos.

  \item Respecto a la aplicación web, aunque se ha tratado de que contase
  con un diseño web adaptable siempre se podría mejorar y reestructurar de
  acuerdo alguna de las metodologías usadas en el diseño de la experiencia del
  usuario.

  \item Además, se podrían incrementar las formas de interacción con la placa.
  Usando la interfaz de línea de comandos, un programa de ordenador con
  interfaz gráfica o mediante una aplicación para dispositivos móviles.

\end{itemize}


% Genera la bibliografía.
\printbibliography

% Añade el aviso de la licencia.
\clearpage

\mbox{}
\vfill

\begin{figure}[!h]
  \centering
  \includegraphics[width=0.2\textwidth]{ccbyncsa}
\end{figure}

\begin{center}
  Este obra está bajo una
  \href{https://creativecommons.org/licenses/by-nc-sa/4.0/}
    {licencia de Creative Commons Reconocimiento-NoComercial-CompartirIgual 4.0
    Internacional}.
\end{center}

\end{document}
