\capitulo{4}{Técnicas y herramientas}{\label{sec:tecyherra}}
Este capítulo presenta las técnicas metodológicas y las herramientas de
desarrollo usadas durante la realización del proyecto. En algunas de las
técnicas y herramientas existen varias alternativas a utilizar. En este capítulo
se comentan las características principales de cada alternativa y se expone el
porqué de la elección tomada.

\section{Técnicas metodológicas}{\label{sec:tecnicas}}

\subsubsection{Scrum}{\label{sec:scrum}}
Scrum es definido por dos de sus mayores promotores \textcite{schwaber2017}
como:
\begin{quotation}``un marco de trabajo en el que las personas pueden
  abordar problemas flexibles y complejos, mientras se entregan productos creativa y
  productivamente del mayor valor posible''
\end{quotation}.

Desde el punto de vista del desarrollo del \extranjerismo{software}, trabajar en
un producto en toda su extensión puede atraer problemas debidos, por ejemplo, a
cambios en los requisitos de los clientes. Para solventar este problema, Scrum
enuncia la realización del trabajo de forma ligera, iterativa e incremental,
permitiendo que el desarrollo pueda responder mejor ante imprevistos.

Scrum propone grupos de trabajo donde sus integrantes tienen diferentes roles y
responsabilidades. El grupo sigue un flujo de trabajo que gira entorno al
concepto de \extranjerismo{sprint}, definido como la unidad básica de
desarrollo. Antes de cada \extranjerismo{sprint} se planifica su duración
temporal, se definen la tareas a realizar en él y cuando ya está en marcha se
revisa diariamente para analizar su evolución. Una vez terminado se comprueban
los resultados obtenidos y se proporcionan los entregables generados.

\section{Herramientas \extranjerismo{hardware} de desarrollo}
  {\label{sec:herramientas}}

\subsubsection{Placa de desarrollo FRDM-K64F}{\label{sec:k64f}}
El uso de placas de desarrollo sirve como toma de contacto al microprocesador y
al resto del \extranjerismo{hardware} que componen un SE. En este proyecto la
placa utilizada es el modelo
\href{https://www.nxp.com/support/developer-resources/evaluation-and-development-boards/freedom-development-boards/mcu-boards/freedom-development-platform-for-kinetis-k64-k63-and-k24-mcus:FRDM-K64F}
{FRDM-K64F}.

\imagen{k64f}{Placa de desarrollo FRDM-K64F \cite{webpage:k64f}}

El microcontrolador (MCU) de esta placa es el \href{https://www.nxp.com/products/processors-and-microcontrollers/arm-based-processors-and-mcus/kinetis-cortex-m-mcus/k-seriesperformancem4/k6x-ethernet/kinetis-k64-120-mhz-256kb-sram-microcontrollers-mcus-based-on-arm-cortex-m4-core:K64_120}
{Kinetis K64}. Este MCU cuenta con el microprocesador \href{https://developer.arm.com/products/processors/cortex-m/cortex-m4}
{Cortex-M4} capaz de funcionar a 120 Mhz. Destaca la capacidad de funcionar en 
modos de energía ultra bajos, su prominente rendimiento y los dispositivos
integrados de conectividad y comunicación como los puertos General Purpose
Input/Output (GPIO), la interfaz Ethernet o los buses de datos serie
I\textsuperscript{2}C o SPI.

La placa K64F ofrece el \extranjerismo{hardware} necesario para poder aprovechar
las características que ofrece el MCU. Cuenta con los pines suficientes para dar 
cabida a los GPIO. La disposición de los conectores es compatible con la
utilizada en placas \href{https://www.arduino.cc/}{Arduino} permitiendo el uso
de multitud de placas de extensión ya existentes.

También presenta un puerto Ethernet que permite conectar la placa a una red
local y, a la postre, a Internet. También cuenta con \extranjerismo{hardware}
extra que posibilita nuevas opciones a la hora de investigar y desarrollar con
la placa. Por ejemplo, puertos USB, para alimentación, depuración y conexión, y
diodos emisor de luz (LED) de colores rojo, verde y azul (RGB).

\subsubsection{Placa de expansión Arduino Basic I/O}{\label{sec:basic-io}}
Como la placa K64F es compatible con las placas de expansión de Arduino se
utiliza la placa \href{https://web.archive.org/web/20160818213905/http://www.msebilbao.com/tienda/product_info.php?cPath=130&products_id=793&osCsid=f967e6ddeaaa2f19050972ff62295a08}
{Arduino Basic I/O} para poder utilizar \extranjerismo{hardware} adicional.

\imagen{basic_io}{Placa de expansión Arduino Basic I/O \cite{webpage:basio-io}}

En concreto, se utilizan sus 4 LED de colores que usados en combinación de la
técnica de modulación de señales pulse-width modulation (PWM) se puede regular
la intensidad del brillo de cada uno de los LED.

\subsubsection{Pantalla de cristal líquido (LCD)}{\label{sec:lcd}}
El disponer de una pantalla permite mostrar información en forma de texto a un
usuario del SE. En particular, esta pantalla puede mostrar 2 líneas de texto
de 16 caracteres cada una. Gracias a esta pantalla el usuario puede
conocer ciertos datos acerca del estado de inicialización de la placa, o puede
usarla para mostrar sus propios mensajes.

Cabe destacar que la pantalla usa una conexión de tipo paralela, necesitando
hasta 16 pines para la transmisión de datos, la alimentación, la iluminación y
el control de la transmisión. Como forma de simplificar su uso la pantalla
incorpora un módulo de comunicación I\textsuperscript{2}C, de tal manera que
solo es necesario el empleo de 4 pines.

\subsubsection{Accesorios extra}{\label{sec:extras}}
Para realizar el montaje de las placas y la pantalla se utilizan varios
accesorios extra.

Las placas de pruebas \cite{webpage:placa-pruebas} son unos tableros con
orificios que permiten conectar otras placas, componentes electrónicos,
conectores y cables. Su orificios están conectados eléctricamente, siguiendo un
patrón de bloques y líneas, permitiendo así la conexión entre los elementos del
sistema a montar. La rapidez y facilidad del montaje y desmontaje de los
elementos facilita el prototipado de sistemas antes de su fabricación.

Con el fin de conectar los pines de los componentes se usan varios cables
puente \cite{webpage:cable-puente}. Algunos conectados directamente a las
placas, otros conectados a las placas de pruebas. Para evitar confusiones los
cables son de colores, de tal manera que es posible distinguir la función que
realiza cada uno.

Por último, como el objetivo del proyecto es realizar un SE que utilice los
protocolos TCP/IP, es necesario conectar la placa K64F con un cable Ethernet
a un determinado equipamiento de acceso a la red, por ejemplo,
un \extranjerismo{switch}.
